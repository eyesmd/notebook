
\subsection{Trie}

\impls{\href{https://github.com/eyesmd/wonderbook/blob/master/strings/trie.cpp}{Wonder}}

\subsection{Hashing}

\impls{\href{https://github.com/eyesmd/wonderbook/blob/master/strings/hashing.cpp}{Wonder} | \href{https://github.com/kth-competitive-programming/kactl/blob/master/content/strings/Hashing.h}{Kth} | \href{https://github.com/e-maxx-eng/e-maxx-eng/blob/48d3922a64c4b9268beb3acf30ee294f7bfefbce/src/string/string-hashing.md}{CPA} }

Dado un hash polinomial:

\begin{equation*}
h(s) = \sum_{i=0}^n s_i \, p^i
\end{equation*}

\begin{itemize}
  \item El hash al agregar un caracter al final es $h(s + c) = h(s) + c * p^{|s|}$
  \item El hash al agregar un caracter al principio es $h(c + s) = c + h(s) * p$
  \item El hash al quitar caracteres se puede calcular despejando $h(s)$
\end{itemize}


\subsection{KMP}

\impls{\href{https://github.com/JonSeijo/jonnotebook/blob/master/codigo/string/kmp.cpp}{Jonno} | \href{https://github.com/kth-competitive-programming/kactl/blob/89fd4e30dcd9e19d3723bd34e74cd46be3fa87c4/content/strings/KMP.h}{Kth}}

Un \textit{borde} es un substring \textbf{estricto} que es prefijo y sufijo a la vez.

Para todo string $s$ y borde no trivial $b$, $b[1..n)$ es borde de $s[1..n)$. Luego, existe algun borde de $s[1..n)$ que puede extenderse a $b$. Por contrarecíproco, si no existe ninguno, $b$ no es borde de $s$.

\textbf{Ejemplo:}

\begin{tabular}{ccccccccc}
    b & a & d & d & b & a & d & b & a \\
    0 & 0 & 0 & 0 & 1 & 2 & 3 & 1 & 2
\end{tabular}
 
Por ende, podemos conseguir todos los bordes de $s$ de manera recursiva iterando sobre los bordes de $s[1..n)$.

Además, vale lo siguiente:

\begin{lema}
    Sea M el borde máximo de S. Entonces, para todo P distinto de S, P borde de S si y solo si P borde de M.
\end{lema}

Entonces, para calcular todos los bordes de un string basta con calcular el máximo borde de todos sus prefijos. Con esta info, todos los bordes de un string $s$ serían $mb[mb[...[mb[s]]...]]$.

\textbf{Ejemplo:} \textbf{malu\textit{ma}}\textit{luma} $\to$ \textbf{ma}lu\textit{ma} $\to$ ma
