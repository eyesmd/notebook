
\subsection{Teoría de juegos}

\subsubsection*{Juegos combinatorios}

Los juegos combinatorios son juegos, de \textbf{dos jugadores} que
\textbf{alternan movimientos}, con \textbf{información perfecta} y que
necesariamente \textbf{terminan con la victoria de uno de los participantes}.
Los juegos combinatorios imparciales son aquellos en donde cada jugador tiene
disponible el \textbf{mismo conjunto de jugadas} en cada configuración posible
del juego. Se dice que utilizan reglas \textit{normales} si el última en jugar
gana, y \textit{misere} de lo contrario. De estos juegos son de los que vamos a
hablar en esta sección.

Veamos para qué juegos conocidos aplican estas definiciones. El poker no lo es
porque tiene azar. La batalla naval tampoco porque se esconde información del
contrincante. El tatetí tampoco porque puede terminar en empate. Las damas, el
ajedrez y el go sí son juegos combinatorios, pero a pesar de que lo parezcan, no
son imparciales. Dada una disposición particular del tablero, cada jugador tiene
distintos movimientos, pues cada jugador tiene sus propias piezas. Para que
fuesen imparciales, un jugador debería poder mover también las piezas de su
contrincante. Al final es dificil encontrar ejemplos. Los que se ven como casos
introductorios suelen ser el \textit{juego de la reducción} y \textit{Nim}.

Una definición alternativa para juegos combinatorios imparciales sería que son
juegos que pueden representarse con un digrafo progresivamente acotado, o sea,
que desde cualquier nodo terminamos llegando sí o sí a un nodo saliente en una
cantidad finita de pasos (nodo $=$ configuración, arista $=$ movimiento).
Jugando con reglas normales donde el último que juega gana, los nodos
sin aristas serían posiciones perdedoras.

\subsubsection*{Posiciones P y N}

Una característica importante para los juegos combinatorios es que para cada
nodo puedo decir quien gana si los jugadores juegan de forma perfecta. Dado un
nodo, decimos que cada nodo es $N$ si gana el primer jugador y $P$ si gana el
segundo.

Una estrategia para resolver un juego es arrancar de los nodos
terminales e inducir para atrás quien gana (un nodo no terminal es $P$ si y solo
si todos sus sucesores son $N$). O sea, si en un juego encontramos cierta
propiedad que cuando se cumple para una configuración, no se cumple para ningún
movimiento del contricante, pero la podemos hacer valer de vuelta en el
siguiente movimiento, dichas configuraciones son buenos candidato para ser un
nodos $N$ (lo que falta es que los nodos terminales sean $N$ y que validen la
propiedad) [ejemplo: Nim de 2 pilas, en donde la estrategia ganadora es jugar lo
que jugó el contricante pero en la otra pila, y donde la propiedad sería que
ambas pilas tengan la misma cantidad de fichas].

\subsubsection*{Nim}

El juego de Nim es un juego donde tengo $n$ pilas con fichas, y un movimiento
legal es elegir una pila y substraer alguna cantidad de fichas. El juego termina
cuando ya no quedan fichas en ninguna pila.

\textbf{Teorema de Bouton:} Una posición $(f_1, f_2, \dots, f_n)$ en Nim es $P$
si y solo si $x_1 \oplus x_2 \oplus \dots \oplus x_n = 0$. 

\todo[inline]{Proof pending}

\todo[inline]{Nim reduction}

\subsubsection*{Sprague-Gundy}

La función de Sprague-Gundy le asigna recursivamente un entero no-negativo a
cada nodo de un juego combinatorio imparcial:

\begin{equation*}
    sg(x) = mex \{ sg(y) : \forall y \in Sucessors(x) \}
\end{equation*}

Para juegos con reglas \textit{normales}, vale que $sg(x) = 0$ si y solo si $x$
es $P$.

\subsubsection*{Sumas de juegos}

Definimos una suma de juegos como el juego que consiste en jugar $n$ juegos
combinatorios independientes, donde el juego termina cuando en ninguno es
posible mover, y gana el último jugador.

\textbf{Teorema de Sprague-Gundy:} Si $sg_i$ es la función es Sprague-Gundy del
juego $G_i$, entonces la función de Sprague-Gundy de la suma de los $G_i$ es
$sg(x_1, \dots, x_n) = sg_1(x_1) \oplus \dots \oplus sg_n(x_n)$.


\todo[inline]{Biblio: https:\/\/www.math.ucla.edu/\~tom\/Game\_Theory\/comb.pdf}

\todo[inline]{Note: El neutro de $\oplus$ es $0$}

\todo[inline]{Game modelling: Marbles (2018 Brazil subregional)}

