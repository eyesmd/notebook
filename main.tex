% --------------------------------------------------
% PREAMBLE -----------------------------------------
% --------------------------------------------------
\documentclass{article}

% DOCUMENT
	\setlength{\parskip}{1.3ex plus 0.2ex minus 0.2ex}
	\setlength{\columnsep}{0.9cm}
\usepackage[landscape,twocolumn]{geometry} % Page layout
    \geometry{
        paper=a4paper,
        tmargin=1.65cm,
        bmargin=1.65cm,
        lmargin=0.9cm,
        rmargin=0.9cm,
        headheight = 0pt,
        headsep = 30pt,
        footskip = 30pt
    }
    
\usepackage{fancyhdr} % Headers
    % * 'fancyhead' and 'fancyfoot' writes to all fielsds
    % * 'leftmark' and 'rightmark' have the info of two top-level sections
    % * 'sectionmark', I don't know
    \pagestyle{fancy}
    \renewcommand{\sectionmark}[1]{\markboth{#1}{}}
    \fancyhead{} 
    \fancyfoot{}
    \rhead{\nouppercase{\leftmark}}
    \cfoot{\thepage / \pageref*{LastPage}}

% ENCODING
\usepackage[utf8]{inputenc}
\usepackage[T1]{fontenc}
\usepackage[spanish]{babel}


% MATH
\usepackage{amsmath} % Has everything
\usepackage{amssymb} % Math symbols
\usepackage{dsfont} % Domain font
\usepackage{amsthm} % Theorems
    \newtheorem*{teorema}{Teorema}
    \newtheorem*{proposicion}{Proposición}
    \newtheorem*{lema}{Lema}
    \newtheorem*{corolario}{Corolario}

% ALGORITHMS
\usepackage{algorithm} % Algorithm floats
    \floatname{algorithm}{Algoritmo}
\usepackage{algorithmicx} % Macro definitions for algorithm typesetting
\usepackage{algpseudocode} % Pseudocode package built with algorithmicx
    \algrenewcommand\algorithmicrequire{\textbf{Requiere}}
    \algrenewcommand\algorithmicensure{\textbf{Asegura}}

% MISC
\usepackage{import}   % Recursive Import
\usepackage{graphicx} % Images
\usepackage{lastpage} % Reference to last page
\usepackage{enumitem} % List control
\usepackage[draft]{todonotes} % Todo's
\usepackage{hyperref} % Hyperlinks
    \hypersetup{
        colorlinks = true,
        urlcolor = blue,
        linkcolor = red,
        citecolor = red
    }
\usepackage{lipsum} % Lorem Ipsum
\usepackage{tabularx} % Table control 
\usepackage{listings} % Source Code
    \lstset{
        language=C++,
        basicstyle=\ttfamily,
        basewidth  = {.45em,0.45em},
        backgroundcolor=\color[rgb]{0.9,0.9,0.9},
        keywordstyle=\color[rgb]{0,0,0.6},
        morekeywords={vector,pair,stack,tuple,queue,map,unordered_map,set,list},
        xleftmargin=4pt,
        xrightmargin=4pt,
        frame=tlbr,
        framesep=4pt,
        framerule=0pt,
        escapechar=\#% char to escape out of listings and back to LaTeX
    }
    \newcommand{\cppfile}[1]{
        \lstinputlisting[language=C++]{#1}
    }

% FONTS
\renewcommand*\rmdefault{ptm} % Times

\newcommand*{\comment}[1]{\hfill\makebox[3.0cm][r]{#1}}%
\newcommand*{\oh}[1]{\comment{$O(#1)$}}%
\newcommand*{\amortizedoh}[1]{\comment{$amortized \,\, O(#1)$}}%
\newcommand*{\cmod}{\, \% \,}%

\newcommand*{\cppalgo}[1]{\comment{$amortized \,\, O(#1)$}}%
    


% ---------------------------------------------------
% MAIN ----------------------------------------------
% ---------------------------------------------------

\begin{document}

\tableofcontents

\pagebreak

\section{Programación}

\subsection{Misc}

\subsubsection*{Input}

\begin{lstlisting}
getline(cin, s); // reads line from input into 's'
\end{lstlisting}

\subsubsection*{Strings}

\begin{lstlisting}
s.substr(i, len); // returns s[i..i+len], or s[i..] if i+len >= sz(s)
\end{lstlisting}

\subsubsection*{Containers}

\begin{lstlisting}
vector<int> v( all(c) );
set<int> s( all(c) );
\end{lstlisting}

\subsubsection*{Pseudo-structs}

La gracia es ya tener definido operadores de comparación (y de print usando defines).
\begin{lstlisting}
typedef tuple<ll, ll, ll> node;
define DAY(node) get<0>(node)
define LEFT(node) get<1>(node)
define HOTEL(node) get<2>(node)
print_with(node, "(" << DAY(x) << ", " << LEFT(x) << ", " << HOTEL(x) << ")");
\end{lstlisting}

\subsubsection*{Permutaciones}

\textit{Next permutation} muta un container a la permutación inmediata superior, y retorna si quedó ordenado. El siguiente snippet utiliza dicha función para \textbf{recorrer en orden} todas las permutaciones de un arreglo/vector \textbf{ya ordenado}:

\begin{lstlisting}
vector<int> v;
sort(v.begin(), v.end());
do {
    // code
} while (next_permutation(v.begin(), v.end()));  #\amortizedohk{n!}#
\end{lstlisting}

La guarda está al final para que el cuerpo del ciclo corra al menos una vez con el vector ordenado (porque la condición de corte es justamente si está ordenado).

\subsubsection*{Repetidos}
\begin{lstlisting}
set<int> s( vec.begin(), vec.end() );
vec.assign( s.begin(), s.end() );
\end{lstlisting}

\subsubsection*{Punteros}

\begin{itemize}
    \item ¡Los punteros son considerados \textit{random-access iterators}! Esto implica que todos los algoritmos de la STL los podemos utilizar sobre arrays (por ejemplo, \textit{sort(arr, arr+n)}).
    \item Los arrays 'decaen' a punteros en casi todas sus operaciones (son excepciones \textit{sizeof} y \textit{\&}). Un ejemplo notable es cuando son pasados como argumentos de funciones, razón por la cual los programas siguen tipando, aún cuando le pasamos un array a una función que espera un puntero (obs: sin embargo, sí fallaría si pasásemos un array de arrays).
\end{itemize}

\subsubsection*{Recordatorios}

\begin{itemize}
    \item ¡Para usar un \textit{unordered\_set} hace falta implementar una función de hash para pares!
    \item \textit{std::find} para sets es O(n) porque es genérico, lo que es O(log n) es \textit{std::set::find}
    \item El operador [] para maps crea un elemento si no lo encuentra, \textit{at} en cambio tira una excepción
    \item ¡El operador módulo devuelve negativos!
\end{itemize}

\pagebreak
\subsection{Bitmasks}

\subsubsection*{Operaciones básicas}

\begin{tabularx}{\textwidth/2-80pt}{X X}
	Conjunto con $n$ elementos & \lstinline{int x = (1 << n) - 1;} \footnote{No utilizamos el más sencillo $-1$ porque al borrar todos los elementos no nos quedaría $0$.}\\
	Pertenencia del $i$-esimo elemento & \lstinline{if (x & 1 << i)} \\
	Cardinal & \lstinline{__builtin_popcount(x)} \\
    Agregar $i$-esimo elemento & \lstinline{x |= 1 << i} \\
    Borrar $i$-esimo elemento & \lstinline{x &= ~(1 << i)} \\
    Recorrer subconjuntos & \lstinline{forn(x, 1 << n)} \\
\end{tabularx}

\subsubsection*{Operaciones de conjunto}

\begin{tabularx}{\textwidth/2-80pt}{X X}
    Complemento & \lstinline{((1 << n) - 1) ^ x)} \\
    Unión & \lstinline{x | y} \\
    Intersección & \lstinline{x & y} \\
    Diferencia & \lstinline{x & ~y} \\
    Diferencia Simétrica & \lstinline{x ^ y} \\
\end{tabularx}

\subsubsection*{Truquitos}

\begin{tabularx}{\textwidth/2-80pt}{X X}
	Potencia de dos inmediata inferior & \lstinline{1 << (31 - __builtin_clz(n))} \\
	Recorrer subconjuntos de subconjuntos en $O(n^3)$ (salvo vacio) & \lstinline{for (int x = y; x > 0; x = (y & (x-1)))} \\
\end{tabularx}

\subsubsection*{Comentarios}
\begin{itemize}
	\item Los operadores \& y | tienen menor precedencia que los operadores de comparación, con lo cual \lstinline{x & 3 == 1} se interpreta como \lstinline{x & (3 == 1)}. ¡Ojo con eso!
	\item Los operadores \textit{builtin} tienen versiones para \lstinline{long} y \lstinline{long long}, (\lstinline{__builtin_clzll(n)}) ¡Cuidado con pasarles el tipo incorrecto!
	\item Los operadores de bits no están completamente definidos sobre enteros con signo. Para código portable y bien escrito, es mejor utilizar tipos sin signo. Dicho eso, en los jueces no explota nada si utilizamos ints. El único cuidado especial es realizar shifts sobre números negativos, por resultar en \textit{undefined behaviour}, que se puede evitar dejando en $0$ el bit más significativo.
\end{itemize}
\pagebreak

\section{Estructuras de Datos}
\section{Data Structures}

La Sparse Table no es dinámica porque tiene mucha redundancia. Alternativamente, esa redudancia es la que le permite devolver queries en O(1) con operaciones idempotentes.

\textit{State: Unverified}
\cppfile{structures/rmq.cpp}

Segment Tree con mínimo puede ser utilizado usado para encontrar la primera ocurrencia de un elemento que cumpla con cierto predicado

\pagebreak

\section{Strings}
\section{Strings}

\subsection*{Bordes}

Un \textit{borde} es un substring \textbf{estricto} que es prefijo y sufijo a la vez.

La aplicación será vista más tarde. A continuación se encuentra el desarrollo que nos va a permitir calcular bordes eficientemente.

Para todo string $s$ y borde no trivial $b$, $b[1..n)$ es borde de $s[1..n)$. Luego, existe algun borde de $s[1..n)$ que puede extenderse a $b$. Por contrarecíproco, si no existe ninguno, $b$ no es borde de $s$.

\todo[inline]{¿Cuándo es que un borde no máximo puede extenderse a borde del siguiente?}

\textbf{Ejemplos:}

\begin{tabular}{ccccccccccc}
    a & b & r & a & c & a & d & a & b & r & a \\
    0 & 0 & 0 & 1 & 0 & 1 & 0 & 1 & 2 & 2 & 3
\end{tabular}

\begin{tabular}{ccccccccc}
    b & a & d & d & b & a & d & b & a \\
    0 & 0 & 0 & 0 & 1 & 2 & 3 & 1 & 2
\end{tabular}
 
Esto da la pauta de como conseguir todos los bordes de $s$ con DP, iterando sobre los bordes de $s[1..n)$ y verificando si pueden extenderse o no. Sin embargo, para cada prefijo tendría que calcular todos los bordes, y tendriamos que guardar $\mathds{O}(n^2)$ bordes. Necesitamos una mejor forma de codificar nuestra información. Para esto vamos a aprovechar la relación que tienen los bordes de un string:

\begin{lema}
    Si A es borde de B y B es borde de C, entonces A es borde de C.
\end{lema}

\begin{lema}
    Dado un par de bordes de C, el menor es borde del mayor.
\end{lema}

\begin{corolario}
    Si $|A| < |B|$ y B es borde de C, entonces A borde de B si y solo si A borde de C
\end{corolario}

\begin{corolario}
    Si M es el borde máximo de S, entonces para todo P distinto de S, P borde de S si y solo si P borde de M
\end{corolario}

\textbf{Ejemplo:} \textbf{malu\textit{ma}}\textit{luma} $\to$ \textbf{ma}lu\textit{ma} $\to$ ma

Lo que conseguimos con esto es darnos cuenta de que para calcular todos los bordes de un string basta con calcular el máximo borde de todos sus prefijos. Calculando un array con esta información, todos los bordes de un string $s$ serían $mb[mb[...[mb[|s|]]...]]$.

\todo[inline]{I mean, is that all there is?}

\pagebreak

\section{Geometria}
\section{Geometria}



\cppfile{geometry/pto.cpp}

Representación elegida de un segmento: paramétrica ($c + tv \quad c, v \in \mathds{R}^2$)

\cppfile{geometry/inter.cpp}

\pagebreak

\section{Matemática}
\subsection*{Álgebra}

\begin{equation*}
    a^2 + bx + c = 0 \; \Longleftrightarrow \; x = \frac{-b \pm \sqrt{b^2 - 4ac}}{2a}
\end{equation*}
\pagebreak

\subsection{Aritmética modular}

\subsubsection*{Propiedades}

Si $a_1 \equiv b_1 \bmod m$ y $a_2 \equiv b_2 \bmod m$, entonces: 
\begin{itemize}
	\item $a_1 + a_2 \equiv b_1 + b_2 \mod m$
	\item $a_1 - a_2 \equiv b_1 - b_2 \mod m$
	\item $a_1 a_2 \equiv b_1 b_2 \mod m$
	\item $\frac{a_1}{a_2} \equiv b_1 b_2^{-1} \mod m$ \quad (siendo $a_2$ y $m$ coprimos)
	\item $a_1^k \equiv b_1^k \mod m$
	\item $a^{p-2} a \equiv 1 \mod p $ \quad (p primo)
\end{itemize}

Con estas propiedades podemos justificar correctitud al aplicar módulo mientras operamos.

\subsubsection*{Lemas}

\begin{itemize}
	\item Teorema de Euler: $a^{\varphi{(n)}} \equiv 1 \bmod n$
	\item Pequeño teorema de Fermat: $a^{p-1} \equiv 1 \bmod p$ (con $p$ primo que no divide a $a$).
\end{itemize}

\subsubsection*{Potencia modular}

El siguiente algoritmo calcula $a^b$ en módulo de manera eficiente:

\cppfile{math/modulo/expmod.cpp}

Podemos pensar que el algoritmo está recorriendo $a$ en binario y por cada $k$-esima posición con un $1$, acumulamos en el resultado $a^{2^k}$. En otras palabra, estamos calculando el resultado para las potencias de dos que conforman al número $b$, y combinando todo en nuestro resultado final.

El algoritmo puede generalizarse para calcular $a \circ a \circ \dots \circ a$ con $\circ$ asociativa, y tal que el operador módulo sea distributivo con respecto a la misma.

\cppfile{math/modulo/opmod.cpp}

\pagebreak
\subsection*{Teoría de juegos}

\subsubsection*{Juegos combinatorios}

Los juegos combinatorios son juegos, de \textbf{dos jugadores} que
\textbf{alternan movimientos}, con \textbf{información perfecta} y que
necesariamente \textbf{terminan con la victoria de uno de los participantes}.
Los juegos combinatorios imparciales son aquellos en donde cada jugador tiene
disponible el \textbf{mismo conjunto de jugadas} en cada configuración posible
del juego. Se dice que utilizan reglas \textit{normales} si el última en jugar
gana, y \textit{misere} de lo contrario. De estos juegos son de los que vamos a
hablar en esta sección.

Veamos para qué juegos conocidos aplican estas definiciones. El poker no lo es
porque tiene azar. La batalla naval tampoco porque se esconde información del
contrincante. El tatetí tampoco porque puede terminar en empate. Las damas, el
ajedrez y el go sí son juegos combinatorios, pero a pesar de que lo parezcan, no
son imparciales. Dada una disposición particular del tablero, cada jugador tiene
distintos movimientos, pues cada jugador tiene sus propias piezas. Para que
fuesen imparciales, un jugador debería poder mover también las piezas de su
contrincante. Al final es dificil encontrar ejemplos. Los que se ven como casos
introductorios suelen ser el \textit{juego de la reducción} y \textit{Nim}.

Una definición alternativa para juegos combinatorios imparciales sería que son
juegos que pueden representarse con un digrafo progresivamente acotado, o sea,
que desde cualquier nodo terminamos llegando sí o sí a un nodo saliente en una
cantidad finita de pasos (nodo $=$ configuración, arista $=$ movimiento).
Jugando con reglas normales donde el último que juega gana, los nodos
sin aristas serían posiciones perdedoras.

\subsubsection*{Posiciones P y N}

Una característica importante para los juegos combinatorios es que para cada
nodo puedo decir quien gana si los jugadores juegan de forma perfecta. Dado un
nodo, decimos que cada nodo es $N$ si gana el primer jugador y $P$ si gana el
segundo.

Una estrategia para resolver un juego es arrancar de los nodos
terminales e inducir para atrás quien gana (un nodo no terminal es $P$ si y solo
si todos sus sucesores son $N$). O sea, si en un juego encontramos cierta
propiedad que cuando se cumple para una configuración, no se cumple para ningún
movimiento del contricante, pero la podemos hacer valer de vuelta en el
siguiente movimiento, dichas configuraciones son buenos candidato para ser un
nodos $N$ (lo que falta es que los nodos terminales sean $N$ y que validen la
propiedad) [ejemplo: Nim de 2 pilas, en donde la estrategia ganadora es jugar lo
que jugó el contricante pero en la otra pila, y donde la propiedad sería que
ambas pilas tengan la misma cantidad de fichas].

\subsubsection*{Nim}

El juego de Nim es un juego donde tengo $n$ pilas con fichas, y un movimiento
legal es elegir una pila y substraer alguna cantidad de fichas. El juego termina
cuando ya no quedan fichas en ninguna pila.

\textbf{Teorema de Bouton:} Una posición $(f_1, f_2, \dots, f_n)$ en Nim es $P$
si y solo si $x_1 \oplus x_2 \oplus \dots \oplus x_n = 0$. 

\todo[inline]{Proof pending}

\todo[inline]{Nim reduction}

\subsubsection*{Sprague-Gundy}

La función de Sprague-Gundy le asigna recursivamente un entero no-negativo a
cada nodo de un juego combinatorio imparcial:

\begin{equation*}
    sg(x) = mex \{ sg(y) : \forall y \in Sucessors(x) \}
\end{equation*}

Para juegos con reglas \textit{normales}, vale que $sg(x) = 0$ si y solo si $x$
es $P$.

\subsubsection*{Sumas de juegos}

Definimos una suma de juegos como el juego que consiste en jugar $n$ juegos
combinatorios independientes, donde el juego termina cuando en ninguno es
posible mover, y gana el último jugador.

\textbf{Teorema de Sprague-Gundy:} Si $sg_i$ es la función es Sprague-Gundy del
juego $G_i$, entonces la función de Sprague-Gundy de la suma de los $G_i$ es
$sg(x_1, \dots, x_n) = sg_1(x_1) \oplus \dots \oplus sg_n(x_n)$.


\todo[inline]{Biblio: https:\/\/www.math.ucla.edu/\~tom\/Game\_Theory\/comb.pdf}

\todo[inline]{Note: El neutro de $\oplus$ es $0$}

\todo[inline]{Game modelling: Marbles (2018 Brazil subregional)}


\pagebreak
%\input{math/primes/primes.tex}
%\pagebreak

\section{Miscelaneas}
\import{general/}{general.tex}
\pagebreak

\listoftodos[TODOs]

\end{document}
