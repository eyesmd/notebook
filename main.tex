% --------------------------------------------------
% PREAMBLE -----------------------------------------
% --------------------------------------------------
\documentclass{article}

% DOCUMENT
    \setlength{\parskip}{1.3ex plus 0.2ex minus 0.2ex}
\usepackage[landscape,twocolumn]{geometry} % Page layout
    \geometry{
        paper=a4paper,
        tmargin=1.5cm,
        bmargin=1.5cm,
        lmargin=1cm,
        rmargin=1cm,
        headheight = 0pt,
        headsep = 25pt,
        footskip = 25pt
    }
\usepackage{fancyhdr} % Headers
    % * 'fancyhead' and 'fancyfoot' writes to all fielsds
    % * 'leftmark' and 'rightmark' have the info of two top-level sections
    % * 'sectionmark', I don't know
    \pagestyle{fancy}
    \renewcommand{\sectionmark}[1]{\markboth{#1}{}}
    \fancyhead{} 
    \fancyfoot{}
    \rhead{\nouppercase{\leftmark}}
    \cfoot{\thepage / \pageref*{LastPage}}

% ENCODING
\usepackage[utf8]{inputenc}
\usepackage[spanish]{babel}

% MATH
\usepackage{amsmath} % Has everything
\usepackage{amssymb} % Math symbols
\usepackage{dsfont} % Domain font
\usepackage{amsthm} % Theorems
    \newtheorem*{teorema}{Teorema}
    \newtheorem*{proposicion}{Proposición}
    \newtheorem*{lema}{Lema}
    \newtheorem*{corolario}{Corolario}

% ALGORITHMS
\usepackage{algorithm} % Algorithm floats
    \floatname{algorithm}{Algoritmo}
\usepackage{algorithmicx} % Macro definitions for algorithm typesetting
\usepackage{algpseudocode} % Pseudocode package built with algorithmicx
    \algrenewcommand\algorithmicrequire{\textbf{Requiere}}
    \algrenewcommand\algorithmicensure{\textbf{Asegura}}

% MISC
\usepackage{graphicx} % Images
\usepackage{lastpage} % Reference to last page
\usepackage{enumitem} % List control
\usepackage[draft]{todonotes} % Todo's
\usepackage{hyperref} % Hyperlinks
    \hypersetup{
        colorlinks = true,
        urlcolor = blue,
        linkcolor = red,
        citecolor = red
    }
\usepackage{lipsum} % Lorem Ipsum
\usepackage{tabularx} % Table control 
\usepackage{listings} % Source Code
    \lstset{
        language=C++,
        basicstyle=\ttfamily,
        basewidth  = {.45em,0.45em},
        backgroundcolor=\color[rgb]{0.9,0.9,0.9},
        keywordstyle=\color[rgb]{0,0,0.6},
        morekeywords={vector,pair,stack,tuple,queue,map,unordered_map,set,list},
        xleftmargin=4pt,
        xrightmargin=4pt,
        frame=tlbr,
        framesep=4pt,
        framerule=0pt,
    }
    \newcommand{\cppfile}[2][]{
        \lstinputlisting[linerange={#1}]{#2}
    }

% FONTS
\renewcommand*\rmdefault{ptm} % Times



% ---------------------------------------------------
% MAIN ----------------------------------------------
% ---------------------------------------------------

\begin{document}

\tableofcontents


\pagebreak


\section{STL}

{\centering \subsection*{Input/Forward Iterators (Generic)}}
\begin{tabularx}{\textwidth/2-20pt}{c X}
    \textbf{min\_element}(first, last, [comp]) & returns iterator to the (first) minimum \\
    \textbf{accumulate}(first, last, init, [bp]) & returns the result of sequentially applying \textit{binary} operations starting from init \\
    \textbf{count}(first, last, val) & returns the repetitions of val (using $==$)
\end{tabularx}

{\centering \subsection*{Random Access Iterators (Arrays)}}
\begin{tabularx}{\textwidth/2-20pt}{c X}
    \textbf{lower\_bound}(first, last, val, [comp]) & returns iterator to the first element greater or equal than val (higher\_bound is strict) \footnote{If the iterator is random-access, the operation on average is logarithmic, otherwise it is linear} \\
    \textbf{sort}(first, last, [comp$<$]) & sorts elements in ascending order in $O(n \, log \, n)$
\end{tabularx}

{\centering \subsection*{Vectors}}
\begin{tabularx}{\textwidth/2-20pt}{c X}
    \textbf{assign}(first, last) & fills the vector using the range given \\
\end{tabularx}

{\centering \subsection*{Sets and Maps}}
\begin{tabularx}{\textwidth/2-20pt}{c X}
    \textbf{lower\_bound}(val) & returns iterator to the first element greater or equal than val (higher\_bound is strict)
\end{tabularx}

\todo[inline]{generic: insert(need to specify pos in vectors), erase, clear, swap. vectors have push\_back, pop\_back. Queues have push, pop, back, front. Stacks have push, pop, top. vectors have 'front' and 'back'}

\todo[inline]{partition, set\_union, set\_difference, set\_intersection, search, nth\_element(intro select, qsort, median of medians magic), is\_permutation}


\pagebreak


\section{Yapas}

\subsection*{Funciones Grandes}

\newcolumntype{Y}{>{\raggedleft\let\newline\\\arraybackslash\hspace{0pt}}X}
\begin{tabularx}{0.33\textwidth}{YYYYYYYY}
                      & $\mathbf{5}$ & $\mathbf{10}$ & $\mathbf{15}$ & $\mathbf{20}$ & $\mathbf{25}$ & $\mathbf{30}$ \\
     $\mathbf{2^n}$            & $32$  & $10^3$    & $3*10^4$ & $10^6$   & $3*10^7$ & $10^9$ \\
     $\mathbf{3^n}$            & $243$ & $5*10^4$  & $10^7$   & $3*10^9$ & $...$    & $...$ \\
     $\mathbf{\binom{n}{n/2}}$ & $10$  & $252$     & $5*10^4$ & $10^5$   & $10^6$   & $10^8$ \\
     $\mathbf{n!}$             & $120$ & $10^{12}$ & $...$    & $...$    & $...$    & $...$
\end{tabularx}

\subsection*{Límites}
\begin{tabular}{ccc}
    INT\_MAX $= 2e9$ & LLONG\_MAX $= 9e18$ & DBL\_MAX $= 1e37$
\end{tabular}

\subsection*{Lambdas}
\begin{lstlisting}
auto f = [] (int a, int b) { return a * b; };
std::accumulate(v.begin(), v.end(), 1, f);
\end{lstlisting}

\subsection*{Repetidos}
\begin{lstlisting}
set<int> s( vec.begin(), vec.end() );
vec.assign( s.begin(), s.end() );
\end{lstlisting}


\pagebreak


\subsection{RMQ}

La Sparse Table no es dinámica porque tiene mucha redundancia. Alternativamente, esa redudancia es la que le permite devolver queries en $O(1)$ con operaciones idempotentes.

\cppfile{structures/rmq.cpp}

Segment Tree con mínimo puede ser utilizado usado para encontrar la primera ocurrencia de un elemento que cumpla con cierto predicado

\pagebreak

\subsection{Trie}

\impls{\href{https://github.com/eyesmd/wonderbook/blob/master/strings/trie.cpp}{Wonder}}

\subsection{Hashing}

\impls{\href{https://github.com/eyesmd/wonderbook/blob/master/strings/hashing.cpp}{Wonder} | \href{https://github.com/kth-competitive-programming/kactl/blob/master/content/strings/Hashing.h}{Kth} | \href{https://github.com/e-maxx-eng/e-maxx-eng/blob/48d3922a64c4b9268beb3acf30ee294f7bfefbce/src/string/string-hashing.md}{CPA} }

Dado un hash polinomial:

\begin{equation*}
h(s) = \sum_{i=0}^n s_i \, p^i
\end{equation*}

\begin{itemize}
  \item El hash al agregar un caracter al final es $h(s + c) = h(s) + c * p^{|s|}$
  \item El hash al agregar un caracter al principio es $h(c + s) = c + h(s) * p$
  \item El hash al quitar caracteres se puede calcular despejando $h(s)$
\end{itemize}


\subsection{KMP}

\impls{\href{https://github.com/JonSeijo/jonnotebook/blob/master/codigo/string/kmp.cpp}{Jonno} | \href{https://github.com/kth-competitive-programming/kactl/blob/89fd4e30dcd9e19d3723bd34e74cd46be3fa87c4/content/strings/KMP.h}{Kth}}

Un \textit{borde} es un substring \textbf{estricto} que es prefijo y sufijo a la vez.

Para todo string $s$ y borde no trivial $b$, $b[1..n)$ es borde de $s[1..n)$. Luego, existe algun borde de $s[1..n)$ que puede extenderse a $b$. Por contrarecíproco, si no existe ninguno, $b$ no es borde de $s$.

\textbf{Ejemplo:}

\begin{tabular}{ccccccccc}
    b & a & d & d & b & a & d & b & a \\
    0 & 0 & 0 & 0 & 1 & 2 & 3 & 1 & 2
\end{tabular}
 
Por ende, podemos conseguir todos los bordes de $s$ de manera recursiva iterando sobre los bordes de $s[1..n)$.

Además, vale lo siguiente:

\begin{lema}
    Sea M el borde máximo de S. Entonces, para todo P distinto de S, P borde de S si y solo si P borde de M.
\end{lema}

Entonces, para calcular todos los bordes de un string basta con calcular el máximo borde de todos sus prefijos. Con esta info, todos los bordes de un string $s$ serían $mb[mb[...[mb[s]]...]]$.

\textbf{Ejemplo:} \textbf{malu\textit{ma}}\textit{luma} $\to$ \textbf{ma}lu\textit{ma} $\to$ ma

\pagebreak
\section{Geometria}

\textit{State: Unverified}
\cppfile{geometry/pto.cpp}

Representación elegida de un segmento: paramétrica ($c + tv \quad c, v \in \mathds{R}^2$)

\textit{State: Tested}
\cppfile{geometry/inter.cpp}

\pagebreak
\section{Math}

\begin{equation*}
    a^2 + bx + c = 0 \; \Longleftrightarrow \; x = \frac{-b \pm \sqrt{b^2 - 4ac}}{2a}
\end{equation*}

\pagebreak
\section{Tips}

\subsection*{Estrategias Generales}
\begin{itemize}
    \item \textbf{Simplificación:} Resolver una versión simplificada del problema. Suele ser más facil, y dar pauta de en donde se encuentra la dificultad en el problema original. También puede servir para encontrar una manera de construir la solución.
    \item \textbf{Fuerza Bruta:} Resolver el problema de forma pavota. Está bueno para tener ya una resolución posible, para poner mejor en contexto que tan óptima tenga que ser la respuesta.
\end{itemize}

\subsection*{Geometria}
\begin{itemize}
    \item Fijar grados de libertad de ser posible (desplazamiento, rotación)
\end{itemize}

\subsection*{C++}
\begin{itemize}
    \item std::find para sets es O(n), lo que es O(log n) es std::set::find
    \item El operador [] para maps crea un elemento si no lo encuentra, mejor usar find
    \item Módulo de negativos da negativo, ojo con eso
\end{itemize}

\pagebreak

\end{document}
