\section{Tips}

\subsection*{Estrategias Generales}
\begin{itemize}
    \item \textbf{Simplificación:} Resolver una versión simplificada del problema. Suele ser más facil, y dar pauta de en donde se encuentra la dificultad en el problema original. También puede servir para encontrar una manera de construir la solución.
    \item \textbf{Fuerza Bruta:} Resolver el problema de forma pavota. Está bueno para tener ya una resolución posible, para poner mejor en contexto que tan óptima tenga que ser la respuesta.
\end{itemize}

\subsection*{Geometria}
\begin{itemize}
    \item Fijar grados de libertad de ser posible (desplazamiento, rotación)
\end{itemize}

\subsection*{C++}
\begin{itemize}
    \item std::find para sets es O(n), lo que es O(log n) es std::set::find
    \item El operador [] para maps crea un elemento si no lo encuentra, mejor usar find
    \item Módulo de negativos da negativo, ojo con eso
\end{itemize}
