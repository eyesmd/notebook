\subsection{Yapas}

\subsubsection*{Funciones Grandes}

\newcolumntype{Y}{>{\raggedleft\let\newline\\\arraybackslash\hspace{0pt}}X}
\begin{tabularx}{0.33\textwidth}{YYYYYYYY}
                      & $\mathbf{5}$ & $\mathbf{10}$ & $\mathbf{15}$ & $\mathbf{20}$ & $\mathbf{25}$ & $\mathbf{30}$ \\
     $\mathbf{2^n}$            & $32$  & $10^3$    & $3*10^4$ & $10^6$   & $3*10^7$ & $10^9$ \\
     $\mathbf{3^n}$            & $243$ & $5*10^4$  & $10^7$   & $3*10^9$ & $...$    & $...$ \\
     $\mathbf{\binom{n}{n/2}}$ & $10$  & $252$     & $5*10^4$ & $10^5$   & $10^6$   & $10^8$ \\
     $\mathbf{n!}$             & $120$ & $10^{12}$ & $...$    & $...$    & $...$    & $...$
\end{tabularx}

\subsubsection*{Límites}
\begin{tabular}{ccc}
    INT\_MAX $= 2e9$ & LLONG\_MAX $= 9e18$ & DBL\_MAX $= 1e37$
\end{tabular}



\subsection{Estrategias}

\subsubsection*{General}
\begin{itemize}
    \item \textbf{Simplificación:} Resolver una versión simplificada del problema. Suele ser más facil, y dar pauta de en donde se encuentra la dificultad en el problema original. También puede servir para encontrar una manera de construir la solución.
    \item \textbf{Fuerza Bruta:} Resolver el problema de forma pavota. Está bueno para tener ya una resolución posible, para poner mejor en contexto que tan óptima tenga que ser la respuesta.
\end{itemize}

\subsubsection*{Geometria}
\begin{itemize}
    \item Fijar grados de libertad de ser posible (desplazamiento, rotación)
\end{itemize}
