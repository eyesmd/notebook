
\subsection{Estrategias}

\subsubsection*{General}
\begin{itemize}
    \item \textbf{Simplificación:} Resolver una versión simplificada del problema. Suele ser más facil, y dar pauta de en donde se encuentra la dificultad en el problema original. También puede servir para encontrar una manera de construir la solución.
    \item \textbf{Fuerza Bruta:} Resolver el problema de forma pavota. Está bueno para tener ya una resolución posible, y para tener más en claro qué tan óptimo tiene que ser la respuesta.
\end{itemize}

\subsubsection*{Geometria}
\begin{itemize}
    \item Fijar grados de libertad de ser posible (desplazamiento, rotación)
\end{itemize}


\subsubsection*{Notas}
\begin{itemize}
    \item ¡Siempre vale la pena pensar un poco más para tratar de hacer el código más sencillo!
\end{itemize}
